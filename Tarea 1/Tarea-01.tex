\documentclass[10pt,letterpaper,fleqn]{article}

\usepackage[utf8]{inputenc}
\usepackage[spanish,es-nodecimaldot]{babel}
\usepackage{amsmath}
\usepackage{amssymb}
\usepackage{multicol}
\usepackage{graphicx}
\usepackage{mdwlist}

\usepackage[dvipsnames]{xcolor}
\usepackage[most]{tcolorbox}

\usepackage{tabu}

\usepackage{mathtools}

\usepackage[top=1in, bottom=1in, left=1in, right=1in]{geometry}


\begin{document}

\begin{titlepage}
    \centering

    {\scshape\LARGE Universidad Nacional Autónoma de México \par}

    \vspace{1cm}
    {\scshape\Large Facultad de Ciencias\par}
    \vspace{1.5cm}

    \begin{center}
        \includegraphics[scale=.1]{assets/img/logo.png}
    \end{center}

    \vspace{.8 cm}

    {\LARGE Tarea 1: \par}
    {\huge\bfseries Ejercicios \par}

    \vspace{0.5cm}
    \large{\itshape{Luis Erick Montes Garcia}} \small{ - 419004547}\\
    \large{\itshape{Hele Michelle Salazar Zaragoza}} \small{ - 316068895}


    \vfill

    Trabajo presentado como parte del curso de
    \textbf{Matematicas Aplicadas para las Ciencias II}
    impartido por el profesor \textbf{Juan Carlos Balleza}. \par
    \vspace{0.1cm}
    {\large Entrega 1 de Marzo 2019 \par}
    \footnotesize{\textbf{Link al código fuente:} git@github.com:lemg98/Matematicas-Aplicadas-II.git}
\end{titlepage}

    \begin{enumerate}

        \item Ejercicio 1. Sea $\overrightarrow{a} = 3\widehat{i} + 4\widehat{j}
        + 5\widehat{k}$ y $\overrightarrow{b} = \widehat{i} - \widehat{j} +
        \widehat{k}$. Calcule (y además represente gráficamente) cada una de las
        siguientes operaciones:
        %Inician incisos del Ejercicio 1.
        \begin{enumerate}
          %Inciso 1(a)
          \item $\overrightarrow{a} + \overrightarrow{b}$ \\
            $ (3,4,5) + (1,-1,1) \\
              =(3+1,4+(-1),5+1) \\
              =(4,3,6)
            $
            \\
          %Inciso 1(b)
          \item $6 \overrightarrow{a} + 8 \overrightarrow{b}$ \\
            $
              {\bf 6}(3,4,5) + {\bf 8}(1,-1,1) \\
              = (6 \cdot 3, 6 \cdot 4, 6 \cdot 5) + (8 \cdot 1, 8 \cdot -1,
              8 \cdot 1) \\
              = (18,24,30) + (8,-8,8) \\
              = (26,16,70)
            $
            \\
          %Inciso 1(c)
          \item $-2 \overrightarrow{a}$ \\
            $
              {\bf -2}(3,4,5) \\
              = (-2\cdot 3, -2\cdot 4, -2\cdot 5) \\
              = (-6,-8,-10)
            $
            \\
          %Inciso 1(d)
          \item $\overrightarrow{a} \cdot \overrightarrow{b}$
            $
              (3,4,5) \cdot (1,-1,1) \\
              = (3 \cdot 1) + (4 \cdot (-1)) + (5 \cdot 1) \\
              = 3 + (-4) + 5 \\
              = 4
            $
            \\
          %Inciso 1(e)
          \item $\overrightarrow{a} \times \overrightarrow{b}$
            $
            (3\widehat{i}, 4\widehat{j}, 5\widehat{k}) \times
            (\widehat{i}, -\widehat{j},\widehat{k}) \\
            = ((4\cdot 1) - (5\cdot(-1)))i + ((5\cdot 1) - (3\cdot 1))j +
            (3\cdot(-1) - (4\cdot 1))k \\
            = (4+5)i + (5-3)j + (-3-4)k \\
            = 9i + 2j -7k
            $
            \\
        %Terminan incisos del Ejercicio 1.
        \end{enumerate}

        \item Ejercicio 2. Encuentre las ecuaciones de rectas y planos que se piden:
        %Inician los incisos del Ejercicio 2%
        \begin{enumerate}

          \item Recta que pasa por el punto (0,1,0) y su vector de dirección está dado por el vector $3i+k$.\\
          \textbf{Solución:} $(x,y,z) = (0,1,0) + \lambda(3,0,1)=(3\lambda,1,\lambda)$.

          \item La recta que pasas por los puntos $(0,2,-1)$ y $(-3,1,0)$.\\
          \textbf{Solución:} Obtenemos el vector de dirección $v=(0,2,-1)( - (-3,1,0) = (-3,1,-1)$ y decimos $(x,y,z)=(0,2,-1) + \lambda(-3,1,-1) = (-3\lambda,\lambda + 2, -\lambda-1)$.
          
                    \item La $ec.$ del plano perperdicular al vector $<-2,1,2>$ y que pasa por el punto P = (-1,1,3).\\
                    \textbf{Solución:} Obtenemos el punto PX $= (x-(-1),y-1,z-3)$ que debe ser perpendicular al vector normal. Por tanto $(x+1  ,y-1,z-3)\cdot(-2,1,2)=-2x-2 + y-1 + 2z - 6 = 0$. Concluimos que la ecuación es $2x-y-2z+9 = 0$.

        \end{enumerate}

        \item Ejercicio 3. Calcule $\overrightarrow{v} \cdot \overrightarrow{w}$
        para los siguientes vectores:
        %Inician incisos del Ejercicio 2.
        \begin{enumerate}
          %Inciso 2(a)
          \item
          $\overrightarrow{v} = -\widehat{i}+\widehat{j}$ y
          $\overrightarrow{v} = \widehat{k}$ \\
            $\overrightarrow{v} = (-1,1,0)$ y
            $\overrightarrow{v} = (0,0,1)$ \\
            $
              (-1,1,0) \cdot (0,0,1) \\
              = 0
            $
            \\
          %Inciso 3(b)
          \item
          $\overrightarrow{v} = -2\widehat{i} -\widehat{j} + \widehat{k}$ y
          $\overrightarrow{w} = 3\widehat{i} + 2\widehat{j} -2\widehat{k}$
            $
              (-2,-1,1) \cdot (3,2,-2) \\
              = (-2\cdot 3) + (-1\cdot 2) + (1\cdot -2) \\
              = (-6) + (-2) + (-2) \\
              = - 6 - 2 - 2 \\
              = -10
            $
        %Terminan incisos del Ejercicio 2.
        \end{enumerate}

        \item Ejercicio 4. Calcule $\overrightarrow{v}\times\overrightarrow{w}$ para los vectores del ejercicio anterior.\\
        %Inician incisos del Ejercicio 5.
        \begin{enumerate}
          %Inciso 5(a)
          \item $\overrightarrow{v} = (-i,j,0)$ $\overrightarrow{w} = (0,0,k)$\\
          %Inicia matriz
          $\overrightarrow{v} \times \overrightarrow{w} =$
          \begin{vmatrix}
            i & j & k \\
            -1 & 1 & 0 \\
            0 & 0 & 1
          \end{vmatrix}
          $ = $
          \begin{vmatrix}
            1 & 0 \\
            0 & 1
          \end{vmatrix}
          $i - $
          \begin{vmatrix}
            0 & -1 \\
            1 & 0
          \end{vmatrix}
          $j + $
          \begin{vmatrix}
            -1 & 1 \\
            0 & 0
          \end{vmatrix}
          $k$ \\
          $(1-0)i - (0-(-1))j + (0-0)k \\
           = i - j$
          \\
          \item $\overrightarrow{v} = (-2i,-j,k)$ $\overrightarrow{w} =
          (3i,2j,-2k)$ \\
          %Inicia matriz.
          $\overrightarrow{v} \times \overrightarrow{w} =$
          \begin{vmatrix}
             i &  j & k \\
            -2 & -1 & 1 \\
             3 &  2 & -2
          \end{vmatrix}
          $ = $
          \begin{vmatrix}
            -1 & 1 \\
            2 & -2
          \end{vmatrix}
          $i -$
          \begin{vmatrix}
            1 & -2 \\
            -2 & 3
          \end{vmatrix}
          $j +$
          \begin{vmatrix}
            -2 & -1 \\
            3 & 2
          \end{vmatrix}
          $k$ \\
          $(2-2)i - (3-4)j + (-4+3)k \\
            =j-k$
        %Terminan incisos del Ejercicio 5.
        \end{enumerate}


        \item Ejercicio 5. Encuentre el área del paralelogramo generado por los
        vectores del Ejercicio 3. \\
        \begin{itemize}

          \item \textbf{Solución:} Tomamos los resultados del inciso anterior y decimos $a = \sqrt{1^2 + 1^2 + 0^2}=\sqrt{2}u^2$.
          \item \textbf{Solución:} Tomamos los resultados del inciso anterior y decimos $a = \sqrt{0^2 + 1^2 + 1^2}=\sqrt{2}u^2$.
        \end{itemize}

        \item Ejercicio 6
        \item Ejercicio 7. Multiplicar las matrices AB, y también hacer el
        producto BA. ¿Es cierto que AB=BA?\\
        \begin{center}
          %Matriz A.
          {\bf A =}
          \begin{pmatrix}
            3 & 0 & 1 \\
            2 & 0 & 1 \\
            1 & 0 & 1
          \end{pmatrix}
          %Matriz B.
          {\bf B =}
          \begin{pmatrix}
            1 & 0 & 1 \\
            1 & 1 & 1 \\
            0 & 0 & 1
          \end{pmatrix}
          \\
        \end{center}

        \begin{center}
          %Multiplicación AB.
          {\bf AB =}
          \begin{pmatrix}
            3+0+0 & 0+0+0 & 3+0+1 \\
            2+0+0 & 0+0+0 & 2+0+1 \\
            1+0+0 & 0+0+0 & 1+0+1
          \end{pmatrix}
          {\bf =}
          \begin{pmatrix}
            3 & 0 & 4 \\
            1 & 0 & 3 \\
            1 & 0 & 2
          \end{pmatrix}
          \\
        \end{center}

        \begin{center}
          %Multiplicación BA.
          {\bf BA =}
          \begin{pmatrix}
            3+0+1 & 0+0+0 & 1+0+1 \\
            3+1+1 & 0+0+0 & 1+1+1 \\
            0+0+1 & 0+0+0 & 0+0+1
          \end{pmatrix}
          {\bf =}
          \begin{pmatrix}
            4 & 0 & 2 \\
            5 & 0 & 3 \\
            1 & 0 & 1
          \end{pmatrix}
        \end{center}
        $\therefore AB \neq BA$

        \item Ejercicio 8
        \item Ejercicio 9. Encuentre el volumen del paralelepipedo generado por
        los vectores (1, 0, 1), (1, 1, 1), (−3, 2, 0).
        \begin{center}
          $\overrightarrow{a} = (i,0j,k)$, $\overrightarrow{b} = (i,j,k)$ y
          $\overrightarrow{c} = (-3i,2j,0k)$. \\
          $|(a \times b) \cdot c|$
        \end{center}
        \begin{vmatrix}
           i &  j & k \\
           1 & 0 & 1 \\
           1 & 1 & 1
        \end{vmatrix}
        $ = $
        \begin{vmatrix}
          0 & 1 \\
          1 & 1
        \end{vmatrix}
        $i -$
        \begin{vmatrix}
          1 & 1 \\
          1 & 1
        \end{vmatrix}
        $j +$
        \begin{vmatrix}
          1 & 0 \\
          1 & 1
        \end{vmatrix}
        $k$ \\
        $(0-1)i - (1-1)j + (1-0)k \\
         = -i+k \cdot (-3i,2j,0) \\
         = -1(-3)+0+0 \\
         = 3$

        \item Ejercicio 10
        \item Ejercicio 11. El volumen de un tetraedro con aristas concurrentes
        $\overrightarrow{a},\overrightarrow{b},\overrightarrow{c}$ está dado por
        la expresión $V = {1\over 6}\overrightarrow{a} \cdot (\overrightarrow{b}
        \times \overrightarrow{c})$.
        %Incisos del Ejercicio 11.
        \begin{enumerate}
          %Inciso 11(a)
          \item Exprese dicho volumen como un determinante.\\
          $V = {1\over 6}\overrightarrow{a} \cdot |\overrightarrow{b}
          \times \overrightarrow{c}|$
          %Inciso 11(b)
          \item Calcule $V$ cuando $\overrightarrow{a} = \widehat{i} +
          \widehat{j} + \widehat{k}$, $\overrightarrow{b} = \widehat{i} -
          \widehat{j} + \widehat{k}$, $\overrightarrow{c} = \widehat{i} +
          \widehat{j}$ \\
          \begin{vmatrix}
            1 & 1 & 1 \\
            1 & -1 & 1 \\
            1 & 1 & 0
          \end{vmatrix}
          $ = 1(0-1)i - 1(0-1)j + 1(1+1)k = -1+1+2 = 2$ \\
          $V = {1\over 6} \cdot 2 = {2 \over 6} u^3$
        \end{enumerate}
        \item Ejercicio 12
        \item Ejercicio 13. Sea $A = (0, 4), B = (3, 1) y C = (−2, −10)$.
        %Incisos del Ejercicio 13.
        \begin{enumerate}
          %Inciso 13(a)
          \item ¿Cuáles son las coordenadas del vector $\overrightarrow{AB}$, y
          del vector $\overrightarrow{BA}$. \\
          $\overrightarrow{AB} = (3i-0i,1j-4j,0k-0k) = (3i,-3j,0k)$ \\
          $\overrightarrow{BA} = (0i-3i,4j-1j,0k-0k) = (-3i,3j,0k)$.

          %Inciso 13(b)
          \item ¿Cuáles son las coordenadas del vector $\overrightarrow{AC}$, y
          del vector $\overrightarrow{BC}$, y también $\overrightarrow{AC} +
          \overrightarrow{CB}$. \\
          $\overrightarrow{AC} = (-2i-0i,-10j-4j,0k-0k) = (-2i,-14j,0k)$ \\
          $\overrightarrow{BC} = (-2i-3i,-10j-1j,0k-0k) = (-5i,-11j,0k)$ \\
          $\overrightarrow{CB} = (3i+2i,1j+10j,0k-0k) = (5i,11j,0k)$ \\
          $\overrightarrow{AC} + \overrightarrow{CB} = (-2i,-14j,0k) +
          (5i,11j,0k) = (3i,-3j,0k)$

          %Inciso 13(c)
          \item Describa gráficamente por qué $\overrightarrow{AC} +
          \overrightarrow{CB} = \overrightarrow{AB}$

        \end{enumerate}

        \item Ejercicio 14

    \end{enumerate}

\end{document}
