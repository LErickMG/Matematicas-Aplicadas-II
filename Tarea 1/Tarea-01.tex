\documentclass[10pt,letterpaper,fleqn]{article}

\usepackage[utf8]{inputenc}
\usepackage[spanish,es-nodecimaldot]{babel}
\usepackage{amsmath}
\usepackage{amssymb}
\usepackage{multicol}
\usepackage{graphicx}
\usepackage{mdwlist}

\usepackage[dvipsnames]{xcolor}
\usepackage[most]{tcolorbox}

\usepackage{tabu}

\usepackage{mathtools}

\usepackage[top=1in, bottom=1in, left=1in, right=1in]{geometry}


\begin{document}

\begin{titlepage}
    \centering

    {\scshape\LARGE Universidad Nacional Autónoma de México \par}

    \vspace{1cm}
    {\scshape\Large Facultad de Ciencias\par}
    \vspace{1.5cm}

    \begin{center}
        \includegraphics[scale=.1]{assets/img/logo.png}
    \end{center}

    \vspace{.8 cm}

    {\LARGE Tarea 1: \par}
    {\huge\bfseries Ejercicios \par}

    \vspace{0.5cm}
    \large{\itshape{Luis Erick Montes Garcia}} \small{ - 419004547}\\
    \large{\itshape{Hele Michelle Salazar Zaragoza}} \small{ - 316068895}


    \vfill

    Trabajo presentado como parte del curso de
    \textbf{Matematicas Aplicadas para las Ciencias II}
    impartido por el profesor \textbf{Juan Carlos Balleza}. \par
    \vspace{0.1cm}
    {\large Entrega 1 de Marzo 2019 \par}
    \footnotesize{\textbf{Link al código fuente:} git@github.com:lemg98/Matematicas-Aplicadas-II.git}
\end{titlepage}

    \begin{enumerate}

        \item Ejercicio 1. Sea $\overrightarrow{a} = 3\widehat{i} + 4\widehat{j}
        + 5\widehat{k}$ y $\overrightarrow{b} = \widehat{i} - \widehat{j} +
        \widehat{k}$. Calcule (y además represente gráficamente) cada una de las
        siguientes operaciones:
        %Inician incisos del Ejercicio 1.
        \begin{enumerate}
          %Inciso 1(a)
          \item $\overrightarrow{a} + \overrightarrow{b}$ \\
            $ (3,4,5) + (1,-1,1) \\
              =(3+1,4+(-1),5+1) \\
              =(4,3,6)
            $
            \\
          %Inciso 1(b)
          \item $6 \overrightarrow{a} + 8 \overrightarrow{b}$ \\
            $
              {\bf 6}(3,4,5) + {\bf 8}(1,-1,1) \\
              = (6 \cdot 3, 6 \cdot 4, 6 \cdot 5) + (8 \cdot 1, 8 \cdot -1,
              8 \cdot 1) \\
              = (18,24,30) + (8,-8,8) \\
              = (26,16,70)
            $
            \\
          %Inciso 1(c)
          \item $-2 \overrightarrow{a}$ \\
            $
              {\bf -2}(3,4,5) \\
              = (-2\cdot 3, -2\cdot 4, -2\cdot 5) \\
              = (-6,-8,-10)
            $
            \\
          %Inciso 1(d)
          \item $\overrightarrow{a} \cdot \overrightarrow{b}$
            $
              (3,4,5) \cdot (1,-1,1) \\
              = (3 \cdot 1) + (4 \cdot (-1)) + (5 \cdot 1) \\
              = 3 + (-4) + 5 \\
              = 4
            $
            \\
          %Inciso 1(e)
          \item $\overrightarrow{a} \times \overrightarrow{b}$
            $
            (3\widehat{i}, 4\widehat{j}, 5\widehat{k}) \times
            (\widehat{i}, -\widehat{j},\widehat{k}) \\
            = ((4\cdot 1) - (5\cdot(-1)))i + ((5\cdot 1) - (3\cdot 1))j +
            (3\cdot(-1) - (4\cdot 1))k \\
            = (4+5)i + (5-3)j + (-3-4)k \\
            = 9i + 2j -7k
            $
            \\
        %Terminan incisos del Ejercicio 1.
        \end{enumerate}

        \item Ejercicio 2
        \item Ejercicio 3. Calcule $\overrightarrow{v} \cdot \overrightarrow{w}$
        para los siguientes vectores:
        %Inician incisos del Ejercicio 2.
        \begin{enumerate}
          %Inciso 2(a)
          \item
          $\overrightarrow{v} = -\widehat{i}+\widehat{j}$ y
          $\overrightarrow{v} = \widehat{k}$ \\
            $\overrightarrow{v} = (-1,1,0)$ y
            $\overrightarrow{v} = (0,0,1)$ \\
            $
              (-1,1,0) \cdot (0,0,1) \\
              = 0
            $
            \\
          %Inciso 3(b)
          \item
          $\overrightarrow{v} = -2\widehat{i} -\widehat{j} + \widehat{k}$ y
          $\overrightarrow{w} = 3\widehat{i} + 2\widehat{j} -2\widehat{k}$
            $
              (-2,-1,1) \cdot (3,2,-2) \\
              = (-2\cdot 3) + (-1\cdot 2) + (1\cdot -2) \\
              = (-6) + (-2) + (-2) \\
              = - 6 - 2 - 2 \\
              = -10
            $
        %Terminan incisos del Ejercicio 2.
        \end{enumerate}

        \item Ejercicio 4
        \item Ejercicio 5. Encuentre el área del paralelogramo generado por los
        vectores del Ejercicio 3. \\
        %Inician incisos del Ejercicio 5.
        \begin{enumerate}
          %Inciso 5(a)
          \item $\overrightarrow{v} = (-i,j,0)$ $\overrightarrow{w} = (0,0,k)$\\
          %Inicia matriz
          $\overrightarrow{v} \times \overrightarrow{w} =$
          \begin{vmatrix}
            i & j & k \\
            -1 & 1 & 0 \\
            0 & 0 & 1
          \end{vmatrix}
          $ = $
          \begin{vmatrix}
            1 & 0 \\
            0 & 1
          \end{vmatrix}
          $i - $
          \begin{vmatrix}
            -1 & 0 \\
            0 & 1
          \end{vmatrix}
          $j + $
          \begin{vmatrix}
            -1 & 1 \\
            0 & 0
          \end{vmatrix}
          $k$ \\
          $(1-0)i - (-1-0)j + (0-0)k \\
           = i + j$
          \\
          \item $\overrightarrow{v} = (-2i,-j,k)$ $\overrightarrow{w} =
          (3i,2j,-2k)$ \\
          %Inicia matriz.
          $\overrightarrow{v} \times \overrightarrow{w} =$
          \begin{vmatrix}
             i &  j & k \\
            -2 & -1 & 1 \\
             3 &  2 & -2
          \end{vmatrix}
          $ = $
          \begin{vmatrix}
            -1 & 1 \\
            2 & -2
          \end{vmatrix}
          $i -$
          \begin{vmatrix}
            -2 & 1 \\
            3 & -2
          \end{vmatrix}
          $j +$
          \begin{vmatrix}
            -2 & -1 \\
            3 & 2
          \end{vmatrix}
          $k$ \\
          $(2-2)i - (4-3)j + (-4+3)k \\
            =-j-k$
        %Terminan incisos del Ejercicio 5.
        \end{enumerate}


        \item Ejercicio 6
        \item Ejercicio 7
        \item Ejercicio 8
        \item Ejercicio 9
        \item Ejercicio 10
        \item Ejercicio 11
        \item Ejercicio 12
        \item Ejercicio 13
        \item Ejercicio 14

    \end{enumerate}

\end{document}
