\documentclass[10pt,letterpaper,fleqn]{article}

\usepackage[utf8]{inputenc}
\usepackage[spanish,es-nodecimaldot]{babel}
\usepackage{amsmath}
\usepackage{amssymb}
\usepackage{multicol}
\usepackage{graphicx}
\usepackage{mdwlist}

\usepackage[dvipsnames]{xcolor}
\usepackage[most]{tcolorbox}

\usepackage{tabu}

\usepackage{pgfplots}
\pgfplotsset{width=10cm,compat=1.9}

\usepackage{mathtools}
\usepackage{tikz}
\usetikzlibrary{trees,positioning}

\usepackage[top=1in, bottom=1in, left=1in, right=1in]{geometry}


\begin{document}

\begin{titlepage}
    \centering

    {\scshape\LARGE Universidad Nacional Autónoma de México \par}

    \vspace{1cm}
    {\scshape\Large Facultad de Ciencias\par}
    \vspace{1.5cm}

    \begin{center}
        \includegraphics[scale=.1]{assets/img/logo.png}
    \end{center}

    \vspace{.8 cm}

    {\LARGE Tarea 1: \par}
    {\huge\bfseries Ejercicios \par}

    \vspace{0.5cm}
    \large{\itshape{Luis Erick Montes Garcia}} \small{ - 419004547}\\
    \large{\itshape{Hele Michelle Salazar Zaragoza}} \small{ - 316068895}
        

    \vfill

    Trabajo presentado como parte del curso de
    \textbf{Matematicas Aplicadas para las Ciencias II}
    impartido por el profesor \textbf{Juan Carlos Balleza}. \par
    \vspace{0.1cm}
    {\large Entrega 1 de Marzo 2019 \par}
    \footnotesize{\textbf{Link al código fuente:} git@github.com:lemg98/Matematicas-Aplicadas-II.git}
\end{titlepage}

    \begin{enumerate}

        \item Ejercicio 1
        \item Ejercicio 2
        \item Ejercicio 3
        \item Ejercicio 4
        \item Ejercicio 5
        \item Ejercicio 6
        \item Ejercicio 7
        \item Ejercicio 8
        \item Ejercicio 9
        \item Ejercicio 10
        \item Ejercicio 11
        \item Ejercicio 12
        \item Ejercicio 13
        \item Ejercicio 14

    \end{enumerate}    

\end{document}